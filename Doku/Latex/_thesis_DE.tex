%%% File encoding: UTF-8
%%% äöüÄÖÜß  <-- keine deutschen Umlaute hier? UTF-faehigen Editor verwenden!

%%% Magic Comments zum Setzen der korrekten Parameter in kompatiblen IDEs
% !TeX encoding = utf8
% !TeX program = pdflatex 
% !TeX spellcheck = de_DE
% !BIB program = biber

\documentclass[master,german]{hgbthesis}
% Zulässige Optionen in [..]: 
%   Typ der Arbeit: diploma, master (default), bachelor, internship 
%   Hauptsprache: german (default), english
%%%----------------------------------------------------------

\RequirePackage[utf8]{inputenc}		% bei der Verw. von lualatex oder xelatex entfernen!

\graphicspath{{images/}}    % Verzeichnis mit Bildern und Grafiken
\logofile{logo}				% Logo-Datei = images/logo.pdf (\logofile{}, wenn kein Logo gewünscht)
\bibliography{references}  	% Biblatex-Literaturdatei (references.bib)

%%%----------------------------------------------------------
% Angaben für die Titelei (Titelseite, Erklärung etc.)
%%%----------------------------------------------------------

%%% Einträge für ALLE Arbeiten: -----------------------------
\title{Analyse von Log-Files eines Web-Servers mithilfe eines Rechnerverbunds}
\author{Steffen Hafner und Daniel Landler-Gärtner}
\programname{Systemadministration}
\placeofstudy{der Hochschule Ravensburg-Weingarten}
\dateofsubmission{2017}{04}{05}	% {YYYY}{MM}{DD}

%%%----------------------------------------------------------
\begin{document}
%%%----------------------------------------------------------

%%%----------------------------------------------------------
\frontmatter                    % Titelei (röm. Seitenzahlen)
%%%----------------------------------------------------------

\maketitle
\tableofcontents

%\chapter{Einleitung}

Die Verfügbarkeit von Daten hat sich in den vergangenen zehn Jahren drastisch verändert. Die Anzahl verschiedener Datenquellen steigt stetig durch die zunehmende Verbreitung mobiler und internetfähiger Geräte. Dadurch sehen sich Unternehmen heute mit sehr viel größeren Datenmengen konfrontiert. Diese gilt es zu erfassen, zu speichern und auszuwerten. Dabei ist es nicht nur die Datenmenge selbst, die den Unternehmen Probleme bereitet, sondern darüber hinaus auch die Struktur und die Art der Daten, sowie die Geschwindigkeit, mit der sie anfallen. 

Der Begriff Big-Data ist in den letzten Jahren vom bloßen Buzz-Word hin zu einem greifbaren technischen Begriff gereift. Hadoop und NoSQL-Technologien haben maßgeblich zu dieser Evolution beigetragen.		
%\include{thesis_DE/front/abstract}			

%%%----------------------------------------------------------
\mainmatter          % Hauptteil (ab hier arab. Seitenzahlen)
%%%----------------------------------------------------------

\chapter{Einleitung}
\label{cha:Einleitung}

Die Verfügbarkeit von Daten hat sich in den vergangenen zehn Jahren drastisch verändert. Die Anzahl verschiedener Datenquellen steigt stetig durch die zunehmende Verbreitung mobiler und internetfähiger Geräte. Dadurch sehen sich Unternehmen heute mit sehr viel größeren Datenmengen konfrontiert. Diese gilt es zu erfassen, zu speichern und auszuwerten. Dabei ist es nicht nur die Datenmenge selbst, die den Unternehmen Probleme bereitet, sondern darüber hinaus auch die Struktur und die Art der Daten, sowie die Geschwindigkeit, mit der sie anfallen.

Der Begriff Big-Data ist in den letzten Jahren vom reinen Buzz-Word hin zu einem greifbaren technischen Begriff gereift. Big-Data sind Datenmengen, die zu groß für traditionelle Datenbanksysteme sind sowie eine hohe Schnelllebigkeit besitzen. Diese Datenmengen sind entweder unstrukturiert oder semi-strukturiert und enstsrpechen somit nicht den Richtlinien herkömmlicher Datenbanksysteme. Die Herausforderung liegt darin die Daten dennoch zu speichern und zu verarbeiten, damit neue Informationen gewonnen werden können. Mögliche neue Informationen sind z.B. empfohlene Kontakte in sozialen Netzwerken, passende Produktempfehlungen in E-Commerce Lösungen oder Artikelvorschläge auf Nachrichtenseiten. 

Eine treffende Definition von Big-Data lässt sich am Besten durch die drei V veranschaulichen. Volume (Speichergröße und Umfang), velocity (die Geschwindigkeit mit der Datenmengen generiert und transferiert werden) und variety (Bandbreite der Datenquellen)\footnote{Gartner IT Glossary: http://www.gartner.com/it-glossary/big-data}. Diese Definition kann durch value und validity ergänzt werden, welche für einen unternehmerischen Mehrwert und die Sicherstellung der Datenqualität stehen\footnote{Big Data – Fluch oder Segen? – Unternehmen im Spiegel gesellschaftlichen Wandels}.

Auf Grund des hohen Aufwands von Echtzeit-Analysen großer Datenmengen wird Big-Data häufig in Verbindung zu Cluster Computing gebracht. Um Cluster Computing zu realisieren ist ein Framework nötig um die Verarbeitung der Daten auf eine große Anzahl an Computern zu verteilen.

Die Visualisierung der gewonnen Informationen erfordert die Bildung von Korrelationen zwischen den einzelnen Datensätzen, um diese in Abhängigkeit voneinander präsentieren zu können. Dies erfordert, im Gegensatz zu normalisierten Daten von relationalen Datenbanken, bei Plain-Text-Analysen einen erheblichen Mehraufwand. 
%\include{thesis_DE/chapters/grundbegriffe}
\chapter{Problem}
\label{cha:Problem}

Bereits vor dem Aufkommen von Big-Data haben Unternehmen Log-Files zur Gewinnung von Einblicken genutzt. Jedoch ist das Problem, dass durch das exponentielle Wachstum aller Datenquellen die Verwaltung und Analyse der Log-Files zu einer immer größeren Herausforderung wird. Log-Files enthalten eine große Menge an Informationen, wobei nicht alle für den jeweiligen Betreiber von gleicher Bedeutung sind. Zudem liegen die Informationen innerhalb der Log-Files in einem schlecht leserlichen Format vor, weshalb eine Analyse von relevanten Informationen und Korrelationen sehr aufwendig ist.

Serverlogs können abhängig von dem Log-Level der jeweiligen Architektur sehr groß ausfallen, wodurch die manuelle Verwaltung und Analyse nahezu unmöglich wird. Andererseits ist die Analyse von System-Logs notwendig um beispielsweise potenzielle Sicherheitsrisiken und Netzwerkfehler erkennen zu können.

Es gibt prinzipiell zwei Arten von Log-Files:

\begin{enumerate}
\item \textbf{Ereignis-Logs} -- ermöglichen einen umfassenden Überblick über die Funktionalität des Systems sowie allen Komponenten zu einem bestimmten Zeitpunkt.
\item \textbf{Benutzer-Logs} -- ermöglichen einen detailierten Einblick in das Nutzerverhalten wie z.B. auf Webseiten. Durch die Analyse der Benutzer-Logs können genauere Informationen über das Verhalten der Benutzer gewonnen werden als mit gewöhnlichen Webanalyse-Diensten wie Google Analytics oder Omniture.
\end{enumerate}

Ein effizienter und automatisierter Prozess wird benötigt damit schnell und akkurat Muster erkannt werden können sowie die großen Datenmengen der Serverlogs erfolgreich bewältigt werden können. Andererseits laufen Unternehmen Gefahr wertvolle Informationen in der riesen Datenflut zu verlieren und dadurch einen datengestützen Wettbewerbsvorteil zu verlieren.

\pagebreak

\section{Webserver Log-File Analyse}
\label{sec:WebserverLogs}

Dig Logfile-Analyse bezeichnet den Prozess der gezielten Überprüfung und Auswertung eines Logfiles. Durch die Auswertung von Webserver Logs können allgemeine Informationen über das Verhalten und die Aktivitäten der Seitenbesucher gewonnen werden.

Webserver Log-Files enthalten folgende Informationen:

\begin{itemize}
\item IP-Adresse und Hostname
\item Zugriffszeitpunkt
\item Vom User verwendeter Browser
\item Vom User verwendetes OS
\item Herkunftslink bzw. -URL
\item Verwendete Suchmaschine inklusive genutzter Keywords
\item Verweildauer
\item Anzahl aufgerufener Seite
\item Zuletzt geöffnete Seite vor dem Verlassen der Webseite
\end{itemize}

Eines der größten Probleme der Webserver-Logfile-Analyse wird durch das zustandslose HTTP Protokoll verursacht. Durch die separate Behandlung der Anfragen, behandelt der Webserver zwei verschiedene Seitenaufrufe eines Clients als zwei unterschiedliche Instanzen. Wodurch eine Analyse des Nutzerverhaltens deutlich erschwert wird.

Um diesen Problemen entgegen zu wirken gibt es zwei gängige Lösungsmöglichkeiten:

\begin{enumerate}
\item \textbf{Vergabe einer Session-ID:} Die Session-ID ist eine serverseitig generierte ID, die im Browser des Nutzers gespeichert wird. Alle folgenden Anfragen eines Nutzers werden durch die vergebene ID kenntlich gemacht.
\item \textbf{Nutzeridentifikation via IP-Adresse:} Nutzer werden über ihre eindeutige IP-Adresse erkannt und bei allen folgenden Anfragen durch diese identifiziert. Voraussetzung dafür ist die Zustimmung des Nutzers zur Erhebung seiner vollständigen IP-Adresse zu Analysezwecken. Ein weiteres Problem ergibt sich aus der dynamischen Vergabe von IP-Adressen oder durch die mehrfache Nutzung der gleichen IP-Adresse. 
\end{enumerate} 

\section{Log-Files des HRW Webserver}
\label{sec:LogHRW}

Für die Veranschaulichung des genannten Problems verwenden wir die Log-Files des Webservers der Hochschule Ravensburg-Weingarten. Um die Log-Files des HRW-Webservers nutzen zu können, mussten diese aus Datenschutzgründen zunächst anonymisiert werden.

Ein anonymisierter Log-File Eintrag sieht wie folgt aus:

\textit{123.234.12.34 - - [06/Apr/2017:06:24:35 +0200] "GET /c/document\_{}library/get\_{}file}
\textit{?uuid=0b9ef55d-812a-4915-9de1-fe5e7f3a0021\&{}groupId=65432 HTTP/1.1" 200 6021}

\textit{"http://www.hs-weingarten.de/web/willkommen/startseite"  "Mozilla/5.0 (iPad; CPU OS 10/2/1 like Mac OS X) AppleWebKit/602.4.6 (KHTML, like Gecko) Version/10.0 Mobile/14D27 Safari/602.1" }

\pagebreak

\textbf{123.234.12.34} repräsentiert die IP-Adresse des Clients der eine Anfrage an den Webserver gestellt hat. Die IP-Adresse ist durch einen Zufallswert so ersetzt, dass ein eindeutiger Zufallswert für jede vorkommende IP-Adresse verwendet wird. Dabei ist ein Zusammenhang noch erkennbar, allerdings eine Rücküberführung unmöglich.

\textbf{"GET /c/document\_{}library/get\_{}file?uuid=0b9ef55d-812a-4915-9de1-}

\textbf{fe5e7f3a0021\&{}groupId=65432 HTTP/1.1"} Der Request Eintrag enthält die verwendete Zugriffsmethode, die angefragte Ressource sowie die HTTP-Version.

\textbf{200 6021} repräsentiert den Statuscode der Anfrage sowie die Größe des Antwortpakets für den Client.

\textbf{"http://www.hs-weingarten.de/web/willkommen/startseite"} enthälft Informationen über welche Seite der Client auf die angefragte Seite zugegriffen hat.

\textbf{"Mozilla/5.0 (iPad; CPU OS 10/2/1 like Mac OS X) AppleWebKit/602.4.6}

\textbf{(KHTML, like Gecko) Version/10.0 Mobile/14D27 Safari/602.1"} zeigt Informationen die der Browser des Clients über sich selbst berichtet.


\section{Geplante Umsetzung}
\label{sec:GeplanteUmsetzung}

Aufgrund der hohen Komplexität eines realen Computer-Clusters soll in diesem Projekt ein Prototyp entstehen, der ein virtuelles Computer-Cluster simuliert und dadurch die möglichen Funktionen eines realen Computer-Clusters veranschaulicht. Zudem soll der Prototyp einfach portierbar, skalierbar und nutzbar sein.

\begin{figure}[!htb]
	\centering
	\includegraphics[width=1.0\textwidth]{gantt}
	\caption{Gantt-Diagramm}
	\label{img:gantt}
\end{figure} 
\chapter{Anforderungsanalyse}
\label{cha:Anforderungsanalyse}

Die geplanten Anforderungen sind aufgeteilt in \textit{Muss} und \textit{Kann} Kriterien.

\textbf{Muss-Kriterien}

\begin{itemize}
\item Verarbeitung erfolgt in virtuellem Cluster
\item Open-Source Framework für skalierbare und verteilt arbeitende Software
\item Persistente Speicherung der analysierten Daten in einer Datenbank
\item Open-Source Datenbank, die mit gewähltem Framework kompatibel ist
\item Gewonnen Informationen sollen Nutzer in einer Textdatei zur Verfügung stehen
\item Prototyp soll plattformunabhängig und ohne aufwendige Konfiguration/Installation testbar sein
\item Prototyp soll in isolierter Umgebung arbeiten, sodass keine Manipulation an Hostsystem stattfindet
\end{itemize}

\textbf{Kann-Kriterien}

\begin{itemize}
\item Gewonnene Informationen werden auf einer Website dargestellt
\item Automatisierte Ausführung der Analyse-Jobs innerhalb des Clusters
\item Automatisierte Instanziierung/Installation des Prototypen auf anderen Rechnern 
\end{itemize}
\chapter{Lösungsvorschläge}
\label{cha:Lösungsvorschläge}

\section{Framework}
\label{sec:Framwork}

Gesucht ist ein Framework mit dem alle \textit{Muss-Kriterien} bestmöglich umgesetzt werden können. Des Weiteren sollen alle Tasks im Gantt-Diagramm \ref{img:gantt} fristgerecht erledigt werden können. Um das Projekt im genannten Zeitraum durchführen zu können, beschränken wir uns bei der Suche auf die bekanntesten Frameworklösungen auf dem derzeitigen Markt.

Die zwei bekanntesten Framworks für skalierbare, verteilt arbeitende Software im Zusammenhang mit großen Datenmengen sind \textit{Hadoop} und \textit{Spark}. Sowohl Hadoop als auch Spark werden unter Linux entwickelt und verwenden native Linux Libraries. Daher begrenzt sich unsere Auswahl des zu verwendenden Betriebssystems auf Linux Distributionen. Dies erleichtert zum einen das Einrichten und zum anderen die Wartung des Frameworks.

\subsection{Hadoop}
\label{subsec:Hadoop}

Hadoop ist ein Java-Framwork der Apache Software Foundation zum verteilten Speichern von Daten und zu deren parallelen Verarbeitung. Hadoop wird dabei in einem horizontal skalierbaren Cluster betrieben, das auf einfachstem Weg wie gewünscht skaliert werden kann. Große Unternehmen wie Yahoo betreiben so Cluster mit über 4000 Knoten\footnote{Referenzzahlen für Unternehmen, die Hadoop einsetzen: http://wiki.apache.org/hadoop/PoweredBy}. Statt der Anschaffung neuer, schnellerer Hardware (Scale Up) wird beim Betrieb von Hadoop vielmehr die Erweiterung des Clusters (Scale Out) um weitere Knoten empfohlen.

Zu den Basiskomponenten von Hadoop, die bei der Installation mitgeliefert werden gehören:

\begin{itemize}
\item \textbf{Hadoop Distributed File System (HDFS):} Ein über das gesamte Cluster verteiltes Dateisystem zur Speicherung der zu verarbeitenden Daten.
\item \textbf{Map-Reduce:} Ein Programmierframework zur verteilten Verarbeitung von Daten gemäß der zweiphasigen Verarbeitung durch Mapper- und Reducer-Klassen.
\item \textbf{YARN:} Verwaltet die Resourcen eines Clusters dynamisch für verschiedene Jobs. 
\end{itemize} 

Zudem verfügt Hadoop über ein großes Ökosystem, das zahlreiche Technologien enthält, die ergänzend zu den genannten Technologien, installiert werden können.


\begin{figure}[!htb]
	\centering
	\includegraphics[width=0.8\textwidth]{hadoopecosystem}
	\caption{Hadoop-Ökosystem}
	\label{img:hadoopecosystem}
\end{figure}


\subsection{Spark}
\label{subsec:Spark}

Die Daten-Analyse Plattform Spark für clustergestützte Berechnungen wird hauptsächlich für die schnelle Ausführung von Jobs genutzt. Mit Apache Spark können Daten transformiert, fusioniert sowie mathematischen Analysen unterzogen werden. Spark ist darauf ausgelegt die Daten dynamisch im RAM des Server-Clusters zu halten und dort zu verarbeiten. Die sogenannte In-Memory-Technologie gewährleistet eine extrem schnelle Auswertung riesiger Datenmengen.

Die besondere Stärke ist das beinhaltete maschinelle Lernen (Machine Learning) mit den Zusätzen MLib (Machine Learning Bibliothek) sowie SparkR(Direkte Verwendung von R-Bibliotheken unter Spark). Dadurch lassen sich iterative Schleifen sehr gut verarbeiten was eine wichtige Vorrausetzung für Machine Learning Algorithmen darstellt.

\begin{figure}[!htb]
	\centering
	\includegraphics[width=0.5\textwidth]{spark}
	\caption{Apache Spark Framework}
	\label{img:sparkframework}
\end{figure}

\pagebreak

\section{Datenbank}
\label{sec:Datenbank}

\subsection{HBase}
\label{subsec:HBase}

Apache HBase ist eine quelloffene, spaltenorientierte NoSQL-Datenbank, die sich in ihrer Architektur und ihrem Aufbau an Google's \textit{BigTable} orientiert. Eine Besonderheit von HBase ist, dass ein Datensatz beliebig viele Spalten haben kann, auch mehr oder weniger als der vorige oder folgende Datensatz. Diese Eigenschaft hilft bei der schnellen persistenten Speicherung von Daten, da diese zuvor nicht normalisiert werden müssen. HBase ist Teil des Hadoop-Ökosystems \ref{img:hadoopecosystem} und kann daher im \textit{verteilten Modus} ausgeführt werden, in dem sie auf Hadoop aufsetzt und das HDFS nutzt, um ihre Daten darin zu speichern. Der Vorteil beim verteilten Speichern von Daten liegt wie auch beim HDFS darin, besonders große Datenmengen unterzubringen und auf Wunsch zu skalieren, indem man weitere Knoten dem Cluster hinzufügt, wenn Performance oder Speicherkapazitäten knapp werden. Bekannte Unternehmen, die HBase verwenden sind: Adobe, Facebook, Netflix, Spotify, Yahoo! uvm.

\subsection{Cassandra}
\label{subsec:Cassandra}

Apache Cassandra ist ebenfalls eine spaltenorientierte NoSQL-Datenbank, die als skalierbares, ausfallsicheres System für den Umgang mit großen Datenmengen in verteilten Systemen (Clustern) konzipiert wurde. Sie entstand ebenfalls nach dem Vorbild von Google's \textit{BigTable}. Cassandra wird häufig zusammen mit dem Framework Spark verwendet und bildet mit zusätzlichen Technologien den SMACK-Stack, bestehend aus \textbf{S}park, \textbf{M}esos, \textbf{A}kka, \textbf{C}assandra und \textbf{K}afka. Daher wird Cassandra eher mit den genannten Technologien verwendet und harmoniert weniger gut mit Hadoop. Bekannte Unternehmen, die Cassandra verwenden sind: Twitter, Digg und Reddit. Auch Facebook nutzte bis 2011 Cassandra, bis diese durch eine Kombination von HBase und HDFS ersetzt wurde. 

\pagebreak

\section{Virtualisierungssoftware}
\label{sec:Virtualisierung}

\subsection{Docker}
\label{subsec:Docker}

Die Open-Source-Technologie Docker ermöglicht es Anwendungen mithilfe von Betriebssystemvirtualisierung in Containern zu isolieren. Durch die Verwendung von Containern können die Applikationen erstellt, ausgeführt, getestet und verteilt werden. 

Durch das Verpacken von Anwendungen in standardisierte Einheiten erfolgt eine Trennung sowie Verwaltung der auf dem Rechner verwendeten Ressourcen. Diese Einheiten enthalten alle Komponenten die eine Software für die Ausführung benötigt wie z.B. Code, Laufzeitmodul, System-Tools und Systembibliotheken. Dadurch ermöglicht Docker eine schnelle, zuverlässige und einheitliche Bereitstellung von Anwendungen - unabhängig von der Umgebung.

Die Isolation der Anwendungen erfolgt bei Docker auf Software Ebene.

\begin{figure}[!htb]
	\centering
	\includegraphics[width=0.5\textwidth]{Docker}
	\caption{Aufbau Docker}
	\label{img:AufbauDocker}
\end{figure}

\subsection{VirtualBox}
\label{subsec:VirtualBox}

Die Virtualisierungssoftware von Oracle ermöglicht die Installation von virtuellen Maschinen auf einem Host System. Dabei greift VirtualBox auf die Hardware-Ressourcen des Hostsystems zurück und stellt einen Teil dem Gastsystem zur Verfügung. Die Isolation der Virtuellen Maschinen erfolgt auf Hardware Ebene. Jede Virtuelle Maschine besitzt somit ein eigenes Betriebssystem.

\begin{figure}[!htb]
	\centering
	\includegraphics[width=0.4\textwidth]{VirtualBox}
	\caption{Aufbau VirtualBox}
	\label{img:AufbauVirtualBox}
\end{figure}
%\chapter[Lösungsauswahl anhand der Anforderungen]{Lösungsauswahl anhand der Anforderungen}
\label{cha:loesungsauswahl}

\section{Framework}
\label{sec:framework}

Als Framework für die gewünschte Umsetzung wird Hadoop genutzt. Die Kriterien die für Hadoop sprechen sind: gute Skalierbarkeit, Bekanntheit, sowie eine solide Grundausstattung, die bei der Installation des Frameworks bereits enthalten ist. Dazu gehören die bereits genannten Bausteine HDFS, Map-Reduce und YARN. Mit diesen zugehörigen Komponenten ist eine Umsetzung des gewünschten Use-Case bereits ohne zusätzliche Installation von Komponenten umsetzbar. Des Weiteren besitzt Hadoop aufgrund seiner Bekanntheit ein großes Ökosystem mit zahlreichen Technologien, die sich bei Bedarf ohne großen Aufwand hinzufügen lassen. Hadoop ist durch die Verwendung des Map-Reduce Verfahrens auf textbasierte Eingabequellen in Form von Text-Files spezialisiert, was für unseren Use-Case von großer Bedeutung ist, da die Webserver Log-Files in Form von Text-Files vorliegen.

Spark im Vergleich dazu, hat seine Vorteile in der Verarbeitung von Datenströmen. Zudem übernimmt Spark nur ein kleiner Bestandteil der Funktionen die Hadoop als Framwork übernimmt. Spark kann nicht alleine betrieben werden, sondern braucht als Basis ein Hadoop Cluster. Spark wäre somit eine Alternative zur Software, die auf dem Cluster verteilt arbeitet und das Cluster organisiert. Damit müssten bei der Verwendung von Spark noch zusätzliche Komponenten installiert werden, die bei Hadoop bereits enthalten sind. Zusätzlich werden bei der Installation von Spark Technologien mitgeliefert, die bei unserem Use-Case keinen Gebrauch finden würden, wie zum Beispiel machine learning. 

\pagebreak

\section{Datenbank}
\label{sec:datenbank}

Aufgrund der vorigen Wahl des Frameworks wird als Datenbank HBase verwendet. HBase zeigt bereits durch ihren ausgeschriebenen Namen "Hadoop Database" die enge Verknüpfung mit Hadoop. Wie bereits im Schaubild Hadoop-Ökosystem \ref{img:hadoopecosystem} gezeigt wurde, gehört HBase zur großen Sammlung an Technologien, die mit Hadoop bequem verknüpfbar sind. Dadurch lässt sich HBase von Haus aus mit Hadoop kombinieren und auf dem bereits vorhandenen Hadoop Clusters verteilt installieren. Zudem bietet der Aufbau der spaltenorientierten NoSQL-Datenbank mit der Besonderheit, dass Datensätze unabhängig voneinander unterschiedliche Spaltenzahlen haben können, einen Vorteil bei der Speicherung der extrahierten Daten aus den Webserver-Log Files. 

Cassandra unterscheidet sich von ihrer Funktionalität her sehr wenig bis gar nicht von HBase. Der einzige ausschlaggebende Unterschied liegt darin, dass Cassandra zusammen mit Spark verwendet wird und weniger gut mit Hadoop harmoniert.

\section{Virtualisierungssoftware}
\label{sec:virtsoft}

Für den Prototyp zur Analyse von Log-Files wird Docker als Virtualisierungssoftware eingesetzt, aufgrund des äußerst sparsamen Umgang mit Ressourcen sowie den kurzen Startzeiten. Docker-Container weisen eine kurze Startzeit auf, da sie nicht erst das Betriebssystem, Ressourcen und Bibliotheken laden müssen, sondern direkt auf die Komponenten und Daten der Betriebsumgebung zugreifen können. 

Somit erfüllt Docker alle Anforderungen des Projekts.
%\chapter{Umsetzung}
\label{cha:Umsetzung}

\section{Hadoop}
\label{sec:Hadoop}

\subsection{HDFS}
\label{hdfs}

Das HDFS ist Bestandteil des Hadoop Frameworks und erfüllt folgende Anforderungen:

\begin{itemize}
\item Betrieb auf Commodity-Hardware
\item Ausfallsicherheit einzelner Knoten
\item Speicherung und Verarbeitung großer Datenmengen
\item Einfache Skalierbarkeit
\end{itemize}

Ein einziger Masterknoten, genannt Name-Node, verwaltet alle Metadaten des Dateisystems, darunter Verzeichnisstrukturen, Dateien und Dateizugriffe der Clients (Apache Software Foundation, 2013). Parallel dazu existieren mehrere Data-Nodes, die den Speicher verwalten, der den entsprechenden Knoten im Cluster zugeordnet ist. Das HDFS bietet ein Set an Funktionen an, das es erlaubt, Daten in das Dateisystem zu schreiben und daraus zu lesen. Es ist nicht nötig, eine eigene Partition für ein HDFS anzulegen, denn es setzt auf einem existierenden Dateisystem, z.B. dem gängigen ext4 (Fourth Extended Filesystem), auf.

\pagebreak

\subsection{Einrichtung und Konfiguration}
\label{subsec:einrichtungkonfig}

Bei der Installation von Hadoop unterscheidet man hauptsächlich zwischen drei möglichen Konfigurationsarten:

\begin{figure}[!htb]
	\centering
	\includegraphics[width=0.9\textwidth]{hadoopconfigtypes}
	\caption{Hadoop Konfigurationsarten}
	\label{img:hadoopconfigtypes}
\end{figure}

Für unseren Use-Case ist die pseudo distributed Variante die Lösung, die unseren Anforderungen am besten entspricht. Mit dieser Konfiguration lassen sich die Funktionalitäten von Hadoop in einem virtuellen Cluster vollständig umsetzen. Hadoop wird offiziell von der Apache Software Foundation\footnote{Apache Software Foundation: http://hadoop.apache.org/releases.html} für Unix-Systeme zur Verfügung gestellt.


Um das Hadoop Framework auf einem Unix-System zu installieren, muss Java, sowie SSH installiert sein.

Anschließend müssen die Konfigurationsdateien von Hadoop angepasst werden, um die gewünschte Konfiguration umzusetzen. Dabei ist eine Anpassung der folgenden Konfigurationsdateien erforderlich:

\textbf{core-site.xml}

\lstset{language=XML}

\begin{lstlisting}
<configuration>
      <property>
          <name>fs.defaultFS</name>
          <value>hdfs://hadoop:9000</value>
      </property>
 </configuration>
\end{lstlisting}

\textit{fs.defaultFS} in Zeile 3 bestimmt, wo der \textit{Name-Node} zu finden ist. Der \textit{Name-Node} wird mithilfe eines Pfads angegeben. Dabei gibt der Pfad den Rechner-Knoten an, auf dem das HDFS betrieben wird. In diesem Beispiel läuft das HDFS auf dem Rechner mit dem Hostname \textit{hadoop} und der Service ist über den Port \textit{9000} ansprechbar. 

\pagebreak

\textbf{hdfs-site.xml}

\lstset{language=XML}

\begin{lstlisting}
<configuration>
    <property>
        <name>dfs.replication</name>
        <value>1</value>
    </property>
</configuration>
\end{lstlisting}


Durch die Eigenschaft \textit{dfs.replication} wird festgelegt, wie viele Kopien der zu verarbeitenden Daten später auf dem Cluster verteilt werden sollen. Da in diesem Fall nur ein Knoten verwendet wird, kann man lediglich eine Kopie speichern.

\textbf{mapred-site.xml}

\lstset{language=XML}

\begin{lstlisting}
<configuration>
    <property>
        <name>mapreduce.framework.name</name>
        <value>yarn</value>
    </property>
</configuration>
\end{lstlisting}

In der obigen Datei wird bestimmt, dass das neue Map-Reduce-Framework \textit{YARN} benutzt wird.

\textbf{yarn-site.xml}

\lstset{language=XML}

\begin{lstlisting}
<configuration>
	<property>
		<name>yarn.nodemananger.aux-services</name>
		<value>mapreduce_shuffle</value>
	</property>
	<property>
		<name>yarn.nodemanager.aux-services.mapreduce.shuffle.class</name>
		<value>org.apache.hadoop.mapred.ShuffleHandler</value>
	</property>
	<propery>
		<name>yarn.nodemanager.vmen-pmem-ratio</name>
		<value>3</value>
	</property>
	<property>
		<name>yarn.nodemanager.delete.debug-delay-sec</name>
		<value>600</value>
	</property>
</configuration>
\end{lstlisting}

Die Eigenschaften in den Zeilen 3 und 7, legen fest, wie Hadoop später die Verteilung (Mischung) der Knoten im Cluster handhaben wird. Der \textit{yarn.nodemanager.vmem-pmem-ratio} stellt das Verhältnis von physikalischem zu virtuellem Speicher dar. Verwenden wir beispielsweise 4 GB RAM, stehen im Cluster 4 * 3 = 12 GB virtueller Speicher für die Ausführung der Map-Reduce-Jobs oder YARN-Anwendungen zur Verfügung. Die Eigenschaft in Zeile 15 gibt an, wie viel Sekunden die Anwendungsdaten für auf \textit{YARN} laufende Anwendungen bestehen bleiben, bevor sie automatisch gelöscht werden. 


\pagebreak

\subsection{Web-Interface von Hadoop}
\label{subsec:webinterface}

Sobald die Services von Hadoop konfiguriert und erfolgreich gestartet wurden, kann man über die zur Verfügung stehenden Web-Interfaces den aktuellen Status des Clusters und seiner einzelnen Knoten abfragen. Mithilfe der IP-Adresse und dem zugehörigen Port lassen sich folgende Seiten abrufen:

\textbf{Übersicht von Hadoop und dessen Name-Node über den Port \textit{50070}}
Die Seite gibt Informationen zur installierten Hadoop-Version, zum verfügbaren Festplattenspeicher, zur Startzeit des Clusters, sowie den Status der vorhandenen Knoten.

\textbf{Übersicht der laufenden und abgeschlossenen Jobs auf dem Cluster, über den Port  \textit{8088}}
Die Übersicht zeigt zum Einen den Status einzelner Knoten im Cluster und zum Anderen Anwendungen, die im Cluster ausgeführt werden/wurden. Dabei lassen sich Log Files der absolvierten Jobs anschauen, um eventuelle Fehler ausfindig zu machen.

\subsection{Map-Reduce}
\label{subsec:mapreduce}

Im Jahr 2004 veröffentlichten zwei Google-Mitarbeiter ein Paper (Dean et al., 2004) zu einem neuen Ansatz, um große, unstrukturierte Daten anzuzeigen und darin suchen zu können. Aus dem Problem heraus, dass die im Unternehmen gespeicherten Datenmengen zu schnell und zu stark wuchsen, um sie mit herkömmlichen Mitteln verarbeiten zu können, entstand das in dem Paper vorgestellte Programmiermodell \textit{Map-Reduce}. Es beschreibt nicht nur, wie man große Datenmengen durchsucht, auswertet und in Schlüssel-Wert-Paare zusammenfasst, sondern auch, wie man diese sogenannten Map-Reduce-Jobs effizient über ein Cluster auf \textit{Commodity-Hardware} ausführt. Die Arbeitsweise des Algorithmus lässt sich in drei Prozessschritte unterteilen.

\begin{figure}[!htb]
	\centering
	\includegraphics[width=0.9\textwidth]{mapreducephases}
	\caption{Die drei Prozessschritte des Map-Reduce Algorithmus}
	\label{img:mapreducephases}
\end{figure}
%\include{thesis_DE/chapters/fazit}

%%%----------------------------------------------------------
\appendix                                            % Anhang 
%%%----------------------------------------------------------

%\include{thesis_DE/back/anhang_a}	% Technische Ergänzungen
%\include{thesis_DE/back/anhang_b}	% Inhalt der CD-ROM/DVD
%\include{thesis_DE/back/anhang_c}	% Chronologische Liste der Änderungen
%\include{thesis_DE/back/anhang_d}	% Quelltext dieses Dokuments

%%%----------------------------------------------------------
%\MakeBibliography                        % Quellenverzeichnis
%%%----------------------------------------------------------

%%%----------------------------------------------------------
\end{document}
%%%----------------------------------------------------------